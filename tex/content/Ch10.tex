\chapter{小振动}
\problem{已知$n$个函数$\{u_1(t),\dots,u_n(t)\}$线性无关的“充分”条件是其朗斯基行列式 (Wronskian) 非零, 定义为
\[
    \mathcal{W}(u_1,\dots,u_n):=\det \begin{pmatrix}
        u_1 & u_2 & \cdots  & u_n\\
        u_1' & u_2' & \cdots & u_n'\\
        \vdots  & \vdots & \ddots & \vdots \\
        u_1^{(n-1)} & u_2^{(n-1)} & \cdots & u_n^{(n-1)} 
    \end{pmatrix}
\]其中$u^{(i)}$代表对$t$的$i$阶导数.
\begin{enumerate}[label=(\arabic*)]
    \item 证明$\me^{-\mi \omega t}$ 和其复共轭$\me^{+\mi \omega t}$是线性无关的, 即$\mathcal{W}(\me^{-\mi \omega t},\me^{+\mi \omega t})\neq 0$; 
    \item 证明任意复函数$u(t)$及其复共轭的朗斯基行列式$\mathcal{W}(u,u^\star)$只有虚部, 并讨论其非零的条件.
\end{enumerate}}
\begin{solution}
    \begin{enumerate}[label=(\arabic*)]
        \item 不难求得$\mathcal{W}(\me^{-\mi \omega t},\me^{+\mi \omega t})=2\mi \omega\neq 0$
        \item 对于任意复函数$u(t)$,
        \[
            \mathcal{W}(u, u^\star)=\det \begin{pmatrix}
                u & u^\star\\
                \dot{u} & \dot{u^\star}\\ 
            \end{pmatrix}
            =u\dot{u^\star}-(u\dot{u^\star})^\star
            =2\mathrm{Im}(u\dot{u^\star})    
        \]
        可见其只有虚部,非零要求$\mathrm{Im}(u\dot{u^\star})\neq 0$
    \end{enumerate}
\end{solution}

\problem{某单自由度系统, 广义坐标为$q$, 拉格朗日量为$L=\dfrac{1}{2}G(t)\dot{q}^2-\dfrac{1}{2}W(t)q^2$, 其中$G(t)$和$W(t)$都是时间的函数.
\begin{enumerate}[label=(\arabic*)]
    \item 若$q_1(t)$和$q_2(t)$为系统运动方程的任意两个线性无关的特解, 证明其朗斯基行列式$\mathcal{W}(t)=W(q_1(t),q_2(t))$满足形式为$\dot{\mathcal{W}}+f(t)\mathcal{W}=0$的微分方程, 并给出$f(t)$的表达式; 
    \item 根据 (1) 的结果, 分析当$G(t)$和$W(t)$满足什么条件时$\mathcal{W}$为常数.
\end{enumerate}}
\begin{solution}
    \begin{enumerate}[label=(\arabic*)]
        \item 易求得系统运动方程为
        \[
            G(t)\ddot{q}-\dot{G}(t)\dot{q}-W(t){q}=0
        \]
        转化为一阶常微分方程组为:
        \[
            \dot{\mathbf{q}}=\mathcal{A}(t)\mathbf{q}
        \]
        式中
        \[
            \mathcal{A}=\begin{pmatrix}
                0&1\\
                -W/G&-\dot{G}/G
            \end{pmatrix}
        \]
        Wronskian 为
        \[
            \mathcal{W}=\begin{pmatrix}
                q_1&q_2\\
                \dot{q_1}&\dot{q_2}
            \end{pmatrix}
            =q_1\dot{q_2}-q_2\dot{q_1}
        \]
        现计算$\dot{\mathcal{W}}$.由线性常微分方程的Liouville定理,
        \[
            \dot{\mathcal{W}}=\mathrm{tr}(\mathcal{A})\mathcal{W}
        \]
        则$f(t)=-\mathrm{tr}(\mathcal{A})=\frac{\dot{G}(t)}{G(t)}$.
        \item 由于$\mathcal{W}\neq 0$,$\dot{\mathcal{W}}=0$意味着$\dot{G}(t)=0$
    \end{enumerate}
\end{solution}

\problem{待施工}
\begin{solution}
    待施工
\end{solution}

\problem{求习题9.5中系统做小振动的特征频率与简正模式, 并分析简正模式的物理意义.}
\begin{solution}
    待施工
\end{solution}

\problem{求习题9.6中系统做小振动的特征频率与简正模式, 并分析简正模式的物理意义.}
\begin{solution}
    待施工
\end{solution}

\problem{求习题9.7中系统做小振动的特征频率与简正模式, 并分析简正模式的物理意义.}
\begin{solution}
    待施工
\end{solution}