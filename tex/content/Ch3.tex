\chapter{相对论时空观}
\problem{考虑$2$维欧氏空间, 取一般坐标$\{u,v\}$, 与直角坐标关系为$x = x(u,v)$, $y = y(u,v)$. 求$2$维欧氏空间度规在$\{u,v\}$坐标下的形式.}
\begin{solution}
    由线元的定义, 我们有
    \[
        \begin{aligned}
            \dd s^2 &= \left(\frac{\pp x}{\pp u} \dd u + \frac{\pp x}{\pp v} \dd v\right)^2 + \left(\frac{\pp y}{\pp u} \dd u + \frac{\pp y}{\pp v} \dd v\right)^2 \\
            &= \left(\frac{\pp x}{\pp u}\right)^2 \dd u^2 + \left(\frac{\pp x}{\pp v}\right)^2 \dd v^2 + \frac{\pp x}{\pp u} \frac{\pp x}{\pp v} \dd u \dd v + \left(\frac{\pp y}{\pp u}\right)^2 \dd u^2 + \left(\frac{\pp y}{\pp v}\right)^2 \dd v^2 + \frac{\pp y}{\pp u} \frac{\pp y}{\pp v} \dd u \dd v \\
            &= \begin{pmatrix}
                \dd u & \dd v
            \end{pmatrix} \begin{pmatrix}
                \left(\frac{\pp x}{\pp u}\right)^2 & \frac{\pp x}{\pp u} \frac{\pp x}{\pp v} \\
                \frac{\pp x}{\pp v} \frac{\pp y}{\pp u} & \left(\frac{\pp y}{\pp v}\right)^2
            \end{pmatrix} \begin{pmatrix}
                \dd u \\ \dd v
            \end{pmatrix}
        \end{aligned}
    \]
    因此, 度规在$\{u,v\}$坐标下的形式为
    \[
        g_{ij} = \begin{pmatrix}
            \left(\frac{\pp x}{\pp u}\right)^2 & \frac{\pp x}{\pp u} \frac{\pp x}{\pp v} \\
            \frac{\pp x}{\pp v} \frac{\pp y}{\pp u} & \left(\frac{\pp y}{\pp v}\right)^2
        \end{pmatrix}
    \]
\end{solution}

\problem{考虑$3$维欧氏空间, 已知球坐标与直角坐标的关系为$x = r \sin\theta \cos\phi$, $y = r \sin\theta \sin\phi$, $z = r \cos\theta$. 求$3$维欧氏空间度规在球坐标下的形式.}
\begin{solution}
    考虑$3$维欧氏空间中的矢量$\vec{v}$, 由此构造坐标系的坐标基矢
    \[
        \begin{aligned}
            \frac{\pp \vec{v}}{\pp r} &= \sin\theta \cos\phi \,\h{x} + \sin\theta \sin\phi \,\h{y} + \cos\theta \,\h{z} \\
            \frac{\pp \vec{v}}{\pp \theta} &= r \cos\theta \cos\phi \,\h{x} + r \cos\theta \sin\phi \,\h{y} - r \sin\theta \,\h{z} \\
            \frac{\pp \vec{v}}{\pp \phi} &= -r \sin\theta \sin\phi \,\h{x} + r \sin\theta \sin\phi \,\h{y}
        \end{aligned}
    \]
    则线元可以写为
    \[
    \begin{aligned}
        \dd s^2 &= \dd \vec{v} \cdot \dd \vec{v} = g_{ij} \dd u^i \dd u^j\\
        &= \begin{pmatrix}
            \dd r & \dd \theta & \dd \phi
        \end{pmatrix} \begin{pmatrix}
            g_{rr} & g_{r\theta} & g_{\theta\phi} \\
            g_{\theta r} & g_{\theta\theta} & g_{\theta\phi} \\
            g_{\phi r} & g_{\phi\theta} & g_{\phi\phi}
        \end{pmatrix} \begin{pmatrix}
            \dd r \\ \dd \theta \\ \dd \phi
        \end{pmatrix}
    \end{aligned}
    \]
    其中$g_{ij} = g_{ji} = \frac{\pp \vec{v}}{\pp u^i} \cdot \frac{\pp \vec{v}}{\pp u^j} =
    \frac{\pp x}{\pp u^i} \frac{\pp x}{\pp u^j} + \frac{\pp y}{\pp u^i}
    \frac{\pp y}{\pp u^j} + \frac{\pp z}{\pp u^i} \frac{\pp z}{\pp u^j}$.
    
    将$g_{ij}$代入线元, 得
    \[
        \dd s^2 = \begin{pmatrix}
            \dd r & \dd \theta & \dd \phi
        \end{pmatrix} \begin{pmatrix}
            1 & 0 & 0 \\
            0 & r^2 & 0 \\
            0 & 0 & r^2 \sin^2\theta
        \end{pmatrix} \begin{pmatrix}
            \dd r \\ \dd \theta \\ \dd \phi
        \end{pmatrix}
    \]
    即$3$维欧氏空间度规在球坐标下的形式为
    \[
        g_{ij} = \begin{pmatrix}
            1 & 0 & 0 \\
            0 & r^2 & 0 \\
            0 & 0 & r^2 \sin^2\theta
        \end{pmatrix}
    \]
\end{solution}
