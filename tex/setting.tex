\setlength\topmargin{-48pt}
\setlength\headheight{0pt}
\setlength\headsep{25pt}
\setlength\marginparwidth{-20pt}
\setlength\textwidth{7.0in}
\setlength\textheight{9.5in}
\setlength\oddsidemargin{-30pt}
\setlength\evensidemargin{-30pt}


\frenchspacing						% better looking spacing

% Call packages we'll need
\usepackage{CJK,CJKnumb}
%usepackage[english]{babel}			% english
\usepackage{graphicx}				% images
\usepackage{amssymb,amsmath}		% math
\usepackage{multicol}				% three-column layout
\usepackage{url,ctex,tikz}					% clickable links
\usepackage{marvosym}				% symbols
\usepackage{wrapfig}				% wrapping text around figures
\usepackage[T1]{fontenc}			% font encoding
\usepackage{charter} 				% Charter font for main content
\usepackage{datetime}				% custom date
\usepackage{mdframed} 
\usepackage{lipsum}%ダミーの文章を入れる
\usepackage{capt-of}
\usepackage[margin=2cm]{geometry}
\usepackage{tcolorbox}
\usepackage{varwidth}
\usepackage{amsmath}
\usepackage{enumitem}
\usepackage{varwidth}
\usepackage{tikz}
\usepackage{pdfpages}
\usepackage{authblk}
\usepackage{mathrsfs}%\mathscr字体
\usepackage{esint}%环路积分
\usepackage[T1]{fontenc}
\usepackage{lmodern}%微元\dj
\usepackage{booktabs} %三线表
\usepackage{tabularx} %表格自动换行
%\usepackage{titlesec}
%\titleformat{\chapter}{\heiti\huge\bfseries\center}


% 使用 tocloft 宏包控制目录样式
%\usepackage{tocloft}

% 设置目录中条目和页码之间的点
%\renewcommand{\cftdotsep}{1}  % 设置点之间的间隔(1是较小的间隔值,默认9)
%\renewcommand{\cftchapleader}{\cftdotfill{\cftdotsep}}  % 章节标题使用点引导线
%\renewcommand{\cftsecleader}{\cftdotfill{\cftdotsep}}  % 小节标题使用点引导线

\ctexset{
    chapter = {
        format = \centering\Huge\bfseries\heiti,  % 字号和字体,例如:二号字,黑体
        name = {第, 章},           % 章节前后的文字,例:第1章
        number = \chinese{chapter},% 使用中文数字编号
        aftername = \quad          % 章节标题与数字之间的空隙
    }
}


\usetikzlibrary{calc}
\tcbuselibrary{breakable}
\tcbuselibrary{skins}
\tcbuselibrary{xparse}%xparse . sty定义的NewDocumentCommand
\usepackage{setspace}
\linespread{1.3}

\everymath{\displaystyle}


\usepackage{xpinyin}
\definecolor{frameinnercolor}{RGB}{0,112,131}%{249,179,72}


	\newdateformat{mydate}{\monthname[\THEMONTH] \THEYEAR}
\usepackage[pdfpagemode=FullScreen,
			colorlinks=false]{hyperref}	% links and pdf behaviour


%————版面设置————%
\usepackage{fancyhdr}
\renewcommand{\headrulewidth}{0.4pt}
\renewcommand{\headwidth}{\textwidth}
\renewcommand{\footrulewidth}{0pt}

\usepackage{physics}

\usepackage{gbt7714}
\newtheorem{definition}{\hspace{2em}定义}[chapter]
\newtheorem{theorem}{\hspace{2em}定理}[chapter]
\newtheorem{law}{\hspace{2em}定律}[chapter]
\newtheorem{equa}{\hspace{2em}公式}[chapter]
\newtheorem{lemma}{\hspace{2em}引理}[chapter]
\newtheorem{proof}{证明}[chapter]
\newtheorem{proposition}{\hspace{2em}命题}[chapter]
\newtheorem{corollary}{\hspace{2em}推论}[chapter]
\newtheorem{remark}{\hspace{2em}注}[chapter]

\newcommand{\pp}{\partial}%偏导数
\newcommand{\me}{\mathrm e}%自然底数
\newcommand{\mi}{\mathrm i}%虚数单位
\newcommand{\mj}{\mathrm j}%虚数单位
\newcommand{\mk}{\mathrm k}%虚数单位
\newcommand{\dps}{\displaystyle}%行间公式
\newcommand{\degree}{^\circ}%角度
\renewcommand{\bar}{\overline}%平均值
\renewcommand{\vec}{\boldsymbol}%粗体向量
%\renewcommand{\vec}{\overrightarrow}%向量箭头
\newcommand{\h}[1]{\boldsymbol{\hat{#1}}} % basis vector notation


\newcounter{problem}
\setcounter{problem}{0}
\newcommand{\problem}[1]{%
        \refstepcounter{problem}%
        {\bfseries{\thechapter.\theproblem\quad } #1}%
    \vspace{3pt}%
}

\newtheorem{solution}{参考解答}[chapter]